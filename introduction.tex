\section{Introduction}

Comparing to the ``bloodline'' based feudal system in the European world, the bureaucracy system Continental East Asia enabled, in some degree, people to climb to a higher social statutes by attending states exams. However, passing the exam was only the start of the career course of these officials. The paths of their career was decided by the design of promoting system. Also, as the emperors saw the system as a method not only to staff his/her executive and jurisdiction system, but also as to control his clever subordinates, the system was dominantly a rewarding system for the whole society. And, therefore, the mechanism of such system reflects the social reality and the believes of the emperors and the Confucianists. Although this field was studied by many historians, most of current papers still limited themselves to a handful of officials along with their interactions. Their research was rather precious with every relationship well confirmed. But the whole society was much unknown, unless we take the representativeness of these key persons under research as grant. 

Fortunately, the China Biographical Database (CBDB) provided an opportunity of studying of promotion system with a quantitative approach. CBDB collected the information of elite officials across two thousand years. Only in the dynasty of Song, there were more than a thousand officials along with their family members that could be used. Furthermore, CBDB had classified and recorded the interactions of these people which was invaluable for a research on the promotion system as the promotion was found, by previous studies, to be related to the opportunity of promotion in some dynasties. Some valuable attempts had been made on CBDB on the visualise and other analyses about the people in CBDB. However, these attempt either followed a traditional qualitative way to analyse the network around key characters or focused on providing visualisation and other technology infrastructures for building qualitative perceptions. 

Therefore, two choice were available to understood the society in the past. Either we believe that the qualitative result to be representative with no evidence, or, we can use CBDB to generate some result with lower precision but cover a larger amount of people to enable the findings to be statistically representative in respect to the whole social. Hence, we adopted the modern social science style quantitative research method. 

The quantitative research in business world already provided a mature system to capture the humans' behaviour and even to measure social networks. For example, we had linear regression and other machine learning method to evaluate the relationships between different factors. We also had the eigenvector network centrality borrowed from graph theory to be used to represent how well a person was connected. And, these methods enables us, even with lower precision, see a bigger picture. Similar to the qualitative research in social science the previous history research focusing on precisions and details for each historical character and each relationships. These were good studies on a human level, but we want to research the question on a society level. Thus, we admit that the social establishment was not a simple existence and there will never be a research method that can find the objective truth like the details of one human, and, we admit that the research method can never be perfect to unveil the whole picture of the society exactly. Therefore, our aim is firstly, to design an approach that is fairly reliable and can reach the truth as far as we can, while, to make the approach to be able to be fine-tuned by future scholars for a better result. 

Our study chose the Song dynasty for two reasons: first, in the Song dynasty, an official must obtain recommendations from several higher rank officials before he can be promoted to a higher career course (Gai-guan), and, thus, the social relationships of the official were found to be related to the ability of obtain recommendations, and, second, the bio-information of Song dynasty was among the most well built dynasties in CBDB. 

The second part of this paper, we reviewed current research on the ranking and promotion mechanism of the bureaucracy system of Song dynasty. The third part provided the hypotheses. Then the fourth part will describe the method of research and the fifth part will provide the findings. In the six part, we would discuss our findings compared with traditional history research. In the final part, we will draw the conclusion and discuss about the limitation and opportunities for future research. 
